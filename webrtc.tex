\clearpage
\chapter{WebRTC example applications}
In this chapter we'll show some example applications and implementations. These are in a minified version and do not necessarily represent the best practices that should be applied. Nevertheless, there will be sources provided where those best practices can be deducted.

\section{Connect}
This examples shows in a minimal way, how to create a RTC connection. Important in this case is that, the sender creates an offer which is then used to create the \lstinline[basicstyle=\ttfamily\color{black}]|localDescription| for the sender and the \lstinline[basicstyle=\ttfamily\color{black}]|remoteDescription| for the receiver. The receiver on the other hand creates an answer which is used to set the \lstinline[basicstyle=\ttfamily\color{black}]|localDescription| of the receiver and the \lstinline[basicstyle=\ttfamily\color{black}]|remoteDescription| of the sender.

\textit{This example shows a version where the sender and receiver are the same client. For a real world application those code parts would be separated, the offer and the answer would be transferred through a signaling service which is not part of the WebRTC specs. Additionally the ICE Candidates would need to negotiated between the peers. This will be explored in a following section.}

\lstinputlisting[language=JavaScript, style=JavaScript]{examples/miniConnect.js}

\section{Disconnect}
The connection can be closed by simply call the close function of the RTC connection.

\lstinputlisting[language=JavaScript, style=JavaScript]{examples/disconnect.js}

\section{Finding ICE Candidates}


\section{Sending Data}
In this section we will showcase the ability to transfer arbitrary data between peers. In our case we will send text data, but it could be data in any format.

\lstinputlisting[language=JavaScript, style=JavaScript]{examples/sendData.js}

\section{Video chat}

\subsection{Accessing client media}

\subsubsection{Feature check}
First we should check if the current environment supports the needed API's.
\lstinputlisting[language=JavaScript, style=JavaScript]{examples/featureCheck.js}

\subsubsection{Access media}
Then when we have checked if the API's is present we can access client medias, given the user has given his consent.

First we'll add a video channel so the user can see the input from the camera.
\lstinputlisting[language=HTML]{examples/accessMedia.html}
For more details on best practices with video medias https://developers.google.com/web/fundamentals/media/video

Then we use this code to access the media and present it with the video tag.
\lstinputlisting[language=JavaScript, style=JavaScript]{examples/accessMedia.js}
https://developer.mozilla.org/en-US/docs/Web/API/MediaDevices/getUserMedia

\subsubsection{Send Media}
\lstinputlisting[language=JavaScript, style=JavaScript]{examples/sendMedia.js}

