\clearpage
\chapter{Introduction}
\index{WebRTC}{WebRTC} is short for web real-time communication, it is an \index{API}{API} that modern browser support and can be used by web developers to implement a \index{peer-to-peer}{peer-to-peer} communication. It can be used to capture and stream audio and/or video data, as well as to exchange arbitrary data between browsers.

\section{Web Browser Support}
All major browser support \index{WebRTC}{WebRTC} in its newest release. Older versions might not, or only partially, implement this \index{API}{API} so the Adapter.js~\autocite{adapterjs} project should be considered for productive solutions. For detailed information on supported browsers we can use the CanIUse~\autocite{caniuse} site.

\section{Signaling Server}
Although the \index{WebRTC}{WebRTC} is a \index{peer-to-peer}{peer-to-peer} communication \index{API}{API} it can't fully function without a server. It needs a signaling server to resolve the connection between peers. After the peers have established a connection they don't need the signaling server anymore.

\section{Related \index{API}{API}'s}
There are multiple related topics to \index{WebRTC}{WebRTC}. In this section we'll try to give a quick overview over the most important ones.

\subsubsection{Media Capture and Streams \index{API}{API}}
This \index{API}{API} is heavily related to \index{WebRTC}{WebRTC}, it provides support for streaming audio and video data.

\subsubsection{\index{WebSocket}{WebSocket}}
Since \index{WebRTC}{WebRTC} needs to establish a connection to a peer there is a need for a intermediate to create this connection. Often such a signaling server is using \index{WebSocket}{WebSocket} since it provides a bidirectional communication.